\documentclass[twocolumn]{article}

\usepackage[english]{./fav-project-headers/fav-headers}

\title{\fav: \favc}
\author{Aarón Bueno Villares \textit{$<$abv150ci@gmail.com$>$}}
\date{\today}

\begin{document}

\twocolumn[
  \maketitle
]

\tableofcontents

\section{Introduction}
This document is the \favc. \favp will be formalized
here as a project responsible for building the \fav application, a
libre algorithm animation with learning purposes\footnote{The
  term ``libre'', from the Spanish and French languages, is a well-known
  alternative to ``free'', because of ``free'', in english, means both
  ``at no cost'' and ``having liberty''.}. The
project is concerned no just with building this
application but also with ensuring its success, meaning that the
application just is used and has continuity prospects.

\section{Surroundings}
\subsection{History of \fav}
About ten months ago, \fav was
being developed during around two months, but in a pretty much
informal manner, whose outcome was a very simple prototype.

That first \fav development was carried out in order to participate in the
VI edition of the \textit{Concurso universitario del software libre}
(Libre Software Universitary Contest), a spanish
universitary programming competition organized by spanish
universities. That prototype is a simple interprete with a simple
graphic interface. Its operation is very simple: the user has to write
an algorithm in a interpreted language designed to \fav (intermediate
between C and pseudocode), and to execute \fav with this algorithm
(its filename) and its parameters. The execution is carry out, first,
saving the complete evolution of the program state, and then,
visualizing it as a animation: a box to each variable, lineally
organized, and showing iteratively the values of this variables as the
stacked set of states are readed. Due to lack of time, the
project (and the contest) had to be discontinued.

Now it will be reseted more formally by means of project
management issues.

\subsection{Origin of the idea}
The principal motivation for realizing this project is to generate an
useful learning tool, due to I had always been interested in teaching,
popular science, and everything else regarding knowledge and its
popularization. Previosly, in the V edition of the contest above
cited, I made other application related
to popular science. Specifically, an application about cladistic (a branch
of biology), unfortunately without social impact, due to not get
the necessary functionalities for having success in the final users,
but with success in the contest itself, with two awards: absolute in
the local contest of my university (Universidad de Cádiz) and a
special mention on the national chapter
of the contest, and with a lot of new experience and knowledge on my
shoulders. With this new knowledge, above all related to visualization
and graphical interfaces (OpenGL and Qt), it occurd to me casually, remembering
the traditional sorting algorithm animations as that of quicksort or
bubblesort (that somebody can found in youtube, for example), to apply
it to my own field: the computer engineering/science. And so was as
\fav idea was born.

Now, in an Erasmus student capacity I will reset the project in a new
formaler manner, and also I will enter it in the current VII
edition of the \textit{Concurso universitario del software libre} as
extra motivation, as a way of spreading and personal satisfaction.

\subsection{Project guidance}
This project will be achieve following the \pmbok
recomendations (fourth edition, 2008). \pmbok is a ``recognized
standard for the
project management profession'' wrote by a group of specialists in the
project management area and coordinate by PMI, assembling the ``subset of
the project management body of knowledge generally recognized as good
practice''. ``PMBOK'' is the acronym of \textit{Project Management
  Body Of Knowledge} and ``PMI'' of \textit{Project Management
  Institute}. This last group is a not-for-profit professional
organization relates to project management profession, which
establishes standards, researches, educates, works as a certificate
authority and, in general, does all types of services relates to the
project management profession. Thus, the \pmbok is recognized as
a higher quality standard in the project management issue.

Due to the reduced nature of this project, compared with the complex
nature of the projects aims of \pmbok, this practices and areas of
knowledge have to be restricted to the bare essentials.

\subsection{Justification}
Facilitating learning is the key issue to accelerate the professional
formation. Specialization is increasingly important and consumes more
and more time. But it is also important to ensure the quality of this
education. Bibliography related to algorithm visualization
software states the visualization together with textual explanations
improves substantially the retention of the students studying a
certaing algorithm. In the other hand, motivation is the
principal requirement in the success of the learning process. An
application making easier the knowledge can increase the motivation of
students while constitutes an advantage to learning, and thus the
quality of the whole learning process is increased.

Furthermore, among all the currently algorithm animation applications
and other related system, there isn't any system that meets the
thinking traits for \fav. The most modern systems have built-in
algorithms and data structures and can't animate any other algorithm
or data structure. Many system aren't libre, or are so old that have
been abandoned. Other systems which can animate or show somehow any
algorithm, aren't simple enough for a learning tool due to work by
means of calls to system's graphic routines.

Thus, there isn't any application or system satisfying the
described needs, which will be subject of the \favp.

\subsection{Concepts and definitions}
\paragraph{Project and product}
According to \pmbok, a project is a ``temporary endeavor undertaken to
create a unique product, service or result''. That means a project has
a defined beginning and a defined end (the project life cycle), and its outcome
is unique. In this case, \favp has as outcome a unique product,
\fav. The life cycle of the project finishes when it the goals have
been accomplish (the outcome product has been generated keeping the
project requirements and constraints), the project fails or it has
been abandoned. The life cycle of the product finishes when the use of
the product is stopped. That means the product life cycle is a
extension of the project life cycle. Thus, the project aims (of its
life cycle) are different from the product aims (that has a
longer-term life cycle).

\paragraph{Software}
Software has to be distinguish from program and application. A program
is pure code: a set of instrucctions written in a certain language and
saved in a file, or binary code in primary storage being ran by the
operative system. Software is a program, or set of programs, together
with its documentation, configuration files, data and so on, forming
an integrated package offering a service. Lastly, and application is a
software that performs a task for human users. \fav is an
application. Software is a synonym for computer software and software
system. Program is a synonym for computer program. Application is a
synonym for application software. Often all of them are used as
synonyms but this concepts they will be used here with its correct
meanings.

\paragraph{Project charter}
A \chart, just as explained by \pmbok, is a document establishing a
contract with the external organization asking about making a
project. In this project, a \chart is a document establishing a
initial setting-up of a project, not a contract with external
entities. This meaning is closer to the \textit{Project Initiation
  Document}, as defined by PRINCE2 methodology. See the discussion in
\parref{sec:purpose}.

\paragraph{Learning-oriented application}
A learning-oriented application will be distinguished here from
teaching-oriented application. A teaching-oriented application is
defined here as an application for supporting human persons teaching
something to other human person. This excludes self-learning. A
learning-oriented application is in contrast defined here as an
application for supporting human persons can learn, without aditional
restrictions. This second definition is thus more general than the
first one. This definitions are \textit{de facto} definitions for this
project. The use of ``educational software'' is avoided here since
it can include other aspects not related specifically to
``adquisition and delivery of knowledge''.

\paragraph{Software community}
A software community is a group of persons together
with its communication tools who are interested in a certain software,
either as a user, developer or simply supporter. This communications
tools are those that have been created expressly for this software by
its originating developers, irrespective of anything else tool used
additionally by the community. This definition is a \textit{de facto}
definition for this project. For taking a complete software community
isn't enough design communication tools, as well it is neccesary to
assemble a interested group of persons. A complete software community
is an essential requirement for ensuring the software continuity.

\section{Purpose of this charter}
\label{sec:purpose}
The purpose of this charter is to formalize \favp as a real
project and its goals. According to \pmbok, a project chart is,
\textit{inter alia}, a document that
``authorizes a project or phase'' and ``establishes a
partnership between the performing organization and the requesting
organization''. Due to \favp is out of any business
or organization, it is independent of any contract with ``requesting
organizations'', and it isn't even neccesary to sign this document to
formalice it, this document doesn't constitute a project chart
according to \pmbok description.

It should be noted as well that this project is also not formaliced by
this document in a institutional manner, for example, as Final Degree
Project. My origin university (Universidad de Cádiz) and my
destination university as Erasmus (Fachhochshule Würzburg) are just
the institutions in sole charge of this formalization.

Nevertheless, this document is in fact a \chart formalizing \favp
inasmuch as an \textit{internal}\footnote{That means this document is
  only for the project itself and it doesn't establish a contract with
  any external entity.} document specifying its definition,
higher-level goals, requirements and restrictions. Its overriding
objetive is to establish unambiguosly the framework on which this work
will be realized by me as the sole executor of this project. Anyway,
this document will be freely available to any other user or developer
interested in the purposes leading the project and its corresponding
product.

\section{Project description}
\subsection{Project definition}
The name of this project is \favp. \favp is the project undertaken to
create \fav and its software community. \fav is, respectively, a
``libre and learning-oriented algorithm animation application''. The
central service of \fav is to show a graphical animation of the execution
process of any algorithm, including whichever neccesary data structure
for its execution.

This project concerns only with building a first version of \fav, and
not a complete application. That means, for example, \fav will not
contain effective self-learning features. The project concern thus
with building a product with a first set of features for getting a
usefull tool and ensuring its continuity (that implies to develop a
community).

\fav is an acronym for ``Free Algorithm Viewer''\footnote{``Free'' and
not ``libre'' was choosen for the name of the application for similarity
with previous projects.}.

\subsection{Goals}
\label{sec:goals}
The sintetized \favp goal list. All words of this list
have been thoroughly choosed for formalizing the project goals.

\begin{condlist}{G}{5.4cm}
  \conditem{To build an
    application improving algorithm learning by means of
    graphical animations of the process execution of any
    algorithm.}
  \conditem{To create its
    software community and ensure its future survival.}
\end{condlist}

\subsection{Project and product scope}
The project aims to create \fav and its software community. Anything
else work which is not directly-related to this aim is outside of the
project. The product aims to support algorithm learning. Anything else
feature which is not directly-related with algorithm learning is
outside of the product.

That means, for example, features for professional purposes is outside
of the product, and thus its corresponding work is outside of the
project.

\subsection{Requirements}

\paragraph{Project requirements}

\begin{condlist}{RPJ}{5.4cm}
  \conditem{The project has an iterative nature with at most four phases
    to distribute the whole work.}
  \conditem{Each phase has its own project plan with a sort of objetives
    to accomplish.}
  \conditem{The project must be executed following the recomendations of
    the \pmbok.}
\end{condlist}

\paragraph{Product requirements}
\begin{condlist}{RPD}{5.4cm}
  \conditem{The product is object-oriented designed.}
  \conditem{The product source code is readable, well-documented, has a
    unified style, and is functional-decoupled and functional-cohesive.}
  \conditem{The product is designed for internationalization.}
  \conditem{The product processes algorithms written in a language
    intended for human reading.}
  \conditem{The product has an intuitive graphical user interface.}
\end{condlist}

\subsection{Constraints}
\paragraph{Project constraints}
\begin{condlist}{CPJ}{5.4cm}
  \conditem{Every document of this project is licensed under the
    Creative Commons Attribution-ShareAlike License terms.}
  \conditem{The project is five-months long (deadline on March 20,
    2013).}
\end{condlist}

\paragraph{Product constraints}
\begin{condlist}{CPD}{5.4cm}
  \conditem{This product is licensed under the GNU General Public
    License terms.}
  \conditem{The documentation of this product is licensed under the GNU
    General Public License terms.}
  \conditem{The product is implemented through the programming language
    C++.}
  \conditem{The product is multiplataform.}
\end{condlist}

\subsection{Assumptions}
\begin{condlist}{A}{5.4cm}
  \conditem{The english language is the best language to further the
    project and product.}
  \conditem{Setting libre project and product is the best form to further
    them.}
  \conditem{Following \pmbok recommendations is a good form to get
    success in the project.}
\end{condlist}

\section{Deliverables}
Outcomes of this project are as follows:

\begin{description}
  \item[\fav] The \fav software, that includes user and developer
    manual, source code and source code documentation.
  \item[\favp memory] A document describing the work realized. A
    detailed summary of the whole project. This document is a
    requirement for passing the ``Final Degree Project''
    course\footnote{\textit{Proyectos Informáticos} in the University
      of Cádiz (Spain), and \textit{Project Arbeit} in the Hochshule
      für angewandte Wissenchaften Würzburg-Schweinfurt (Germany).}.
\end{description}

\section{Objetives and chronogram}

\begin{figure*}[th!]
    \caption{Preliminary chronology}
    \label{tl:pretl}

    \begin{timeline}{20}{10}{2012}{20}{3}{2013}

      \timeperiod{21}{10}{2012}{20}{12}{2012}{Phase one}
      \timeperiod{21}{12}{2012}{19}{02}{2013}{Phase two}
      \timeperiod{20}{02}{2013}{20}{03}{2013}{Phase three}

      %% First stage
      \event{20}{10}{2012}{Project start-up}{\favc must be
        finished. This event launchs the project execution.}

      \event{5}{11}{2012}{Default animation of numeric values}{Default
        animation of values of algorithms working only with
        numbers feature must be implemented.}

      \event{29}{11}{2012}{Default animation of numeric
        algorithms}{Default animation of values and operations of
        algorithms working only with numbers feature must be
        implemented.}

      \event{7}{12}{2012}{Default animation with any primitive
        type}{Default animation of values and operations of algorithms
        working with any primitive type feature must be implemented.}

      \event{20}{12}{2012}{Default animation of control flow
        statements}{Default animation of values, operations and
        control flow statements of algorithms working with any
        primitive type feature must be implemented.}

      %% Second stage

      \event{19}{02}{2013}{Customizing animations with primitive
        types}{Allowing users to customize animations of
        primitive types, its operations and control flow statements
        feature must be implemented.}

      %% Third stage

      \event{20}{03}{2013}{Data structures}{Allowing users to create
        their own data structures and customizing its animation
        feature must be implemented.}
    \end{timeline}

\end{figure*}

A first group of concrete objetives and timelines will be described
here as a means of further planification orientation. Only functional
features for the product will be described here. Objetives related to
set up a software community are outside of this preliminary
timeline.

The project has (provisionally) three stages (iterations), each stage
with a planning,
execution and monitoring-and-controlling phases. The planning of each
stage (the objectives to accomplish) depends on results of the
previous stage. This project follow thus the spiral software
development process model.

Objectives of the first stage will be more concretly defined in this
chronology, while the other stages will be only defined on its final
result. Figure \numref{tl:pretl} depicts the preliminary chronogram and
objetives.

In short, the first stage takes up implementing the default animation
without data structures, the second stage implementing animation
customization, and the last stage implementing data structures
(built-in and user's data structures, with default and
customizing animations). It will be assumed heaviest stage is that
dedicated to customization (the second stage). Nevertheless, the first
stage will be also a long-term stage due to the uncertainly of a first
contact with the application.

\section{Risks}
Two higher-level risks have been identified:

\begin{itemize}
  \item Trying to create an impossible application: perhaps, it is
    impossible or really heavy to customize any type of algorithm
    animation in a comfortable and usefull manner.
  \item Not getting an initial impact for spreading the application:
    perhaps, meeting with an initial interested group of persons that
    help to spread the application isn't achieve.
\end{itemize}

\section{Reporting framework}
The project development evolution will be well-documented and freely
available on the Internet, by means of a blog, web page to source code
documentation and public repositories. The different communication
channel will be described as follows:

\begin{description}
  \item[\href{http://freealgview.blogspot.com}{\fav blog}] Blog to
    inform about the project and product evolution, included technical
    issues, problems and so on. In this blog will be linked the other
    links decribed below.
  \item[\href{http://peregring-lk.github.com/FreeAlgView}{\fav
      documentation website}] Website to upload the \fav source code
    documentation (Doxygen). This webpage will be of use as official
    website of the product.
  \item[\href{https://github.com/Peregring-lk/FreeAlgView}{\fav
      repository}] Git repository of the \fav source code.
  \item[\href{https://github.com/Peregring-lk/FreeAlgView-project-doc}{\favp
  repository}]  Git repository of the \favp documents source code (the
    documents of the project will be created in \LaTeX).
\end{description}

The project documents (\chart, \plan, etcétera) will be upload in
the \fav documentation website but they will be directly linked from
the blog.
\end{document}
