\documentclass[twocolumn]{article}

\usepackage[english]{./fav-project-headers/fav-headers}

\title{\fav: \favpl stage 1}
\author{Aarón Bueno Villares \textit{$<$abv150ci@gmail.com$>$}}
\datelicense{October 21, 2012}

\setcounter{tocdepth}{2}

\begin{document}

\twocolumn[
  \maketitle
]

\tableofcontents

%% Section introduction
\section{Introduction}
\subsection{Purpose}
This document has two main purposes:

\begin{itemize}
\item To establish the planning to the whole project (see
  \parref{part:st0-plan}), that means, the work framework for all
  stages of the project (let us call it stage 0 plan).
\item To establish the planning of the current stage 1 (see
  \parref{part:st1-plan}).
\end{itemize}

\subsection{Planning guidance}
This planning has been guided by the next list of documents:

\begin{description}
  \item[\pmbok] This guide is the starting point to planning the
    whole project. It is used here as a bibliography about project
    management and general guidance about the managerial work to carry
    out.
  \item[IEEE Std 1074-1997] This standard is used as a information
    source to find the activities to carry out in this project.
\end{description}

This project and its plans will not strictly adhered to these
documents. These documents will be used only as general guidance and
source of knowledge.

%% \subsection{Evolution of the plan}
%% If there are changes in the project plan, it must to add a history
%% table at the beginning of this document (just below the
%% title). Otherwise, that table isn't necessary.

%% It must to add in this table a row for the first version and for each
%% update. It must to inform in this table about the new version number
%% and a short description about the changes made. The format of the
%% version number must be $x.y$, where $x$ denotes version and $y$
%% subversion. There are no specifications about when it has to made a
%% change of version or subversion. This depends on the impact of the
%% changes made. The top row must be the latest updating, and the bottom
%% row the initial version, in chronological order.

\part{\favpl}
\label{part:st0-plan}

%% Section overview
\section{Overview}
\subsection{Purpose}
The purpose of this project plan is to establish the planning of the
work to carry out during the whole project. That means, to establish
the work to carry out throughout all stages of the project and plans
on how to carry out this work to achieve succesfully the goals
described in the \favp.

\subsection{Dependences}
This document constitute a extension of the \favc. The clauses
described here depends thus on the clauses described in the \favc.

\subsection{Planning method}
The method used to plan the project is as follows:

\begin{description}
\item[Identifing risks] To identify all situations which could make the
  project fails.
\item[Identifing at-risk scenes] To identify causes producing these risks.
\item[Identifing requirements] To identify project and product
  requirements to make them resistant to this at-risks scenes.
\item[Designing ``aid plans''] To design plans for
  requirements invisible to the product, that means, those which isn't
  built \textit{inside} the product, but only creating a quality of
  it. These type of requirements require an invisible continuous
  effort to carry out throughout the project.
\end{description}

Any plans will be just now specified, while other plans will be
delegated to later stages of the project, when they are more
meaningful. It is necessary thus to stablish first a dependence
diagram between the activities to detect when it is necessary
develop those plans.

\subsection{Rol of the stage plans}
This project plan defines fundamentally the common work to realice
during all stages of the project. These work have in common that they
relate all to work to non-functional purposes, that means, they are
works to ensure \textit{quality}, but not to develop the functional
requirements of the product.

Thus, the rol of the stage plans is define the work to carry out to
achieve the pure functional requirements specified in the \favc, and
the work to accomplish in a stage is the sum of the functional work of
the stage and the quality work specified in this project plan.

Exceptions are the plans delegated to later project stages, that
are works with quality purposes but defined in concrete stage
plans. Also it is possible in concrete stages to make changes or
extensions in previosly defined plans if this is neccesary.

%\subsection{Changes respect to project managemet}

\section{Quality requirements}
\subsection{Risk identification}
The identified risks, and its causes, in order of priority and dependency
each other, are the following:

\begin{labelist}
\item The project fails.
  \begin{labelist}
    \item Defined objetives not accomplish due to lack of time.
    \item Loss of time due to code rewritting by removal of
      libraries.
  \end{labelist}
\item The product fails.
  \begin{labelist}
    \item Nobody knows the product.
    \item The product doesn't attract new users or developers.
      \begin{labelist}
        \item The information about the product isn't well-defined or
          inaccessible.
        \item Newcommers can't get past the initial obstacle of
          unfamiliarity.
        \item The product doesn't work in the most common plataforms.
        \item The product installation isn't easy.
        \item The product use isn't suitable for newcommers.
        \item The product doesn't work properly.
        \item Developers can't colaborate easily.
      \end{labelist}
    \item Users stop using the product.
      \begin{labelist}
        \item The product doesn't progress.
      \end{labelist}
    \item Developers stop colaborating.
      \begin{labelist}
        \item It is difficult increment the product.
        \item The develop is incoherent.
        \item Lack of new needs, ideas or error reports.
        \item There isn't continuous motivation.
      \end{labelist}
  \end{labelist}
\item Actions to avoid project or product failings, fail.
  \begin{labelist}
    \item The proposed actions aren't correct solutions for this
      problems.
  \end{labelist}
\end{labelist}

\subsection{Requirements}
Only a group of risks are considered in this plan. These risks are
rejected because there is no reason to consider them now. Remainders
risks must be controlled in other stages. Some risks are related or
depend each other, and can be avoided with a only requirement or
plan.

The risks rejected here are the followings: \labelid{B.1},
\labelid{B.2.\{1, 2, 4, 7\}}, \labelid{B.3}, \labelid{B.4.\{2, 3,
  4\}}.

\subsubsection{Requirements for the product}
\label{sssec:prodreq}

\begin{labeledpars}{ProdReq}
  \labeledpar{Documentation for developers} The product source code must be
  written in accordance with readability criteria. The product must
  have a development manual and a code documentation page, both
  freely-available on the Internet.
  \labeledpar{Design} The product design must be designed in
  accordance with extensibility criteria. The product must be
  multiplataform.
  \labeledpar{GUI} To design an interface in accordance with usability
  criteria. The GUI must be designed for internationalization.
\end{labeledpars}

\subsubsection{Requirements for the project}
\begin{labeledpars}{ProjReq}

  \labeledpar{Development plan} The project must have a development
  plan in order to estimate and control better the progress and the
  required time of the project.

  \labeledpar{Importation plan} The project must have an importation
  plan in order to ensure the adjustment of libraries to the product
  needs.

  \labeledpar{Investigation plan} The project must have an
  investigations plan in order to ensure the quality of designs and
  decisions made.

  \labeledpar{Monitoring plan} The project must have a monitoring plan
  in order to improve time estimations.

  \labeledpar{Testing plan} The project must have a testing plan
  in order to ensure the quality of design and implementation of the
  product.

  \labeledpar{Integrated project plan} The project must have an integrated
  project plan in order to know better and estimate the total work to
  carry out.
\end{labeledpars}

\subsection{Risks-requirements matrix}
\label{ssec:rr-matrix}

In this place a matrix relating risks and requirements is shown. The
purpose of this traceability matrix is to detect the source risk of
each requirement. Some risks could need more than one requirement to
avoid it. Similarly, some requirements could concern to more than one
risk. Table \numref{tab:rr-matrix} shows this traceability matrix.

\begin{table*}[th!]
  \caption{Risk-requirements traceability matrix}
  \label{tab:rr-matrix}
  \centering
  \begin{tabular}{|c||c|c|c|c|c|c|c|}
    \hline
    & \mlabelid{A.1} & \mlabelid{A.2} & \mlabelid{B.2.3} &
    \mlabelid{B.2.5} & \mlabelid{B.2.6} & \mlabelid{B.4.1} &
    \mlabelid{C.1} \\
    \hline
    \mlabelid{ProdReq1} & & & & & & \checkmark &\\
    \hline
    \mlabelid{ProdReq2} & & & \checkmark & & & \checkmark &\\
    \hline
    \mlabelid{ProdReq3} & & & & \checkmark & & &\\
    \hline
    \mlabelid{ProjReq1} & \checkmark & & & & & &\\
    \hline
    \mlabelid{ProjReq2} & & \checkmark & & & & &\\
    \hline
    \mlabelid{ProjReq3} & & & & & & & \checkmark\\
    \hline
    \mlabelid{ProjReq4} & \checkmark & & & & & &\\
    \hline
    \mlabelid{ProjReq5} & & & & \checkmark & & &\\
    \hline
    \mlabelid{ProjReq6} & \checkmark & & & & & &\\
    \hline
  \end{tabular}
\end{table*}

\section{Aid plans}
\label{sec:aid-plans}

\subsection{Development plan}
\label{ssec:development-plan}

For simplifying the work to accomplish and improving its measurement,
a development plan and a monitoring plan are established. This clause
describes only the development plan.

\subsubsection{Stages definition}
The whole work must be divided in stages. Each stage generates a
subset of the \fav software features and a subset of the \fav
community features. The last stage generates a final version with all
of its features and satisfying all requirements and constrains defined
in the \favc and in \parref{sssec:prodreq}.

Each stage must identify the features to accomplish and the
requirements to satisfy, and to estimate the neccesary time to carry
out this work.

\subsubsection{Software release numbering system}
\label{sssec:release}
The actual software state at the end of the stage $x$ must be release
with the software version number $0.x$. For example, at the end of the first
stage, the version number is $0.1$, and at the end of the second
stage, $0.2$. However, the last stage produces the \fav version
$1.0$.

\subsubsection{Documentation}
Each stage must be documented in three forms; other plans could add new
clauses:

\paragraph{Stage plans}
Each stage must have its own stage plan. In this stage plan the
selected features to accomplish and the requirements to satisfy must
be enumerated. If new risks are considered, these must be specified in
the project plan and the risks-requirements matrix
(\parref{ssec:rr-matrix}) must be properly extended and
showed. Finally, the stage timeline must be depicted after estimating
the work time.

\paragraph{Blog}
In the project blog, each stage plan must be uploaded and, at the end
of each stage, each release of a new version (\parref{sssec:release})
must be announced, with the description of achievements and video
samples showing the software running.

\paragraph{Software source code repository}
In the software source code repository, a new tag registering the new
version must be inserted. This tag must be well-described with
information about its actual features and limitations.

\subsection{Importation plan}
\label{ssec:importation-plan}
This importation plan contains the requirements and policies about the
use of imported software, mainly libraries. This plan is related to
the investigation plan (\parref{ssec:investigation-plan}) and the
testing plan (\parref{ssec:testing-plan}).

\subsubsection{Imported software requirements}
All imported software must be libre under GPL-compatible
licenses and multiplataform. All imported software must have an active
development and a good documentation. All imported software must be
selected in order to facilitate the development, and they must be thus
well studied before importing it in the \fav development.

\subsubsection{Imported software testing}
In order to avoid selecting software which can later delay the
development, a investigation activity must be carried out each time
that a new library is needed to solve a new implementation problem.

\subsection{Investigation plan}
\label{ssec:investigation-plan}

\subsection{Monitoring plan}
\label{ssec:monitoring-plan}

\subsection{Testing plan}
\label{ssec:testing-plan}

\section{Integrated project plan}
\label{ssec:integrated-plan}

\section{Technical issues}
\subsection{Libraries}
\subsection{Tools}

\part{\favpl stage 1}
\label{part:st1-plan}

\subsection{Purpose}
The purpose of this document is plan the first stage of the
\favp.

\subsection{Dependences}
This document constitute a extension of the \favpl. The clauses
described here depends thus on the clauses described in the \favpl
(see \parref{part:st0-plan}).

\subsection{Deliverables}
This stage 1 has the next deliverables:

\begin{description}
\item[\fav v0.1] To build the \fav version 0.1, that includes to design a
  \fav programming language which manages algorithm working with
  numeric data types, its operations, functions and recursion and
  control flow structures (sequence, iteration and selection
  structures); and to design a default animation for each of this
  language features.
\item[Documentation] To record this stage and to publish on the
  Internet this information, that includes the registered time for
  each realized activity, the features of the achieved version of the
  \fav and its design.
\end{description}

\subsection{Selected requirements}
\subsection{Historial time records}

\section{Work detach}
\subsection{WBS}
\subsection{Sequence activities}
\section{Integrated stage plan}

\section{Summary}
%% %% Section SLCP
%% \section{Project life cycle}

%% %% Section managerial process plans
%% \section{Managerial plans}
%% \subsection{Planning}
%% \subsubsection{Estimation plan}

%% \subsection{Executing}
%% \subsubsection{Work activities}
%% \subsubsection{Schedule allocation}
%% \subsubsection{Resource allocation}

%% \subsection{Control and monitoring}
%% \subsubsection{Schedule control plan}
%% \subsubsection{Reporting plan}
%% \subsubsection{Verification and validation plan}

%% \subsection{Risk management plan}

%% \subsection{Closeout plan}

%% % Section technical process plans
%% \section{Technical plans}
%% \subsection{Process model}
%% \subsection{Methods, tools and technics}
%% \subsubsection{Programming style}
%% \subsection{Infraestructure plan}
%% \subsection{Deployment plan}

%% \part{\favpl stage 1}
%% \label{part:st1-plan}

%% %% Section introduction
%% \section{Introduction}

%% \subsection{Purpose}
%% The purpose of this document is plan the first stage of the
%% \favp.

%% \subsection{Dependences}
%% This document constitute a extension of the \favpl. The clauses
%% described here depends thus on the clauses described in the \favpl
%% (see \parref{part:st0-plan}).

%% \subsection{Deliverables}
%% This stage 1 has the next deliverables:

%% \begin{description}
%% \item[\fav v0.1] To build the \fav version 0.1, that includes to design a
%%   \fav programming language which manages algorithm working with
%%   numeric data types, its operations, functions and recursion and
%%   control flow structures (sequence, iteration and selection
%%   structures); and to design a default animation for each of this
%%   language features.
%% \item[Documentation] To record this stage and to publish on the
%%   Internet this information, that includes the registered time for
%%   each realized activity, the features of the achieved version of the
%%   \fav and its design.
%% \end{description}

%% \section{Managerial process plans}
%% \subsection{Executing}
%% \subsubsection{Work activities}
%% \subsubsection{Schedule allocation}
%% \subsubsection{Resource allocation}

%% \section{Technical process plans}
%% \subsection{Process model}
%% \subsection{Methods, tools and technics}

\end{document}
