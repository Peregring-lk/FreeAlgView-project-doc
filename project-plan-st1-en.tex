\documentclass[twocolumn]{article}

\usepackage[english]{./fav-project-headers/fav-headers}

\title{\fav: \favpl stage 1}
\author{Aarón Bueno Villares \textit{$<$abv150ci@gmail.com$>$}}
\datelicense{October 21, 2012}

\begin{document}

\twocolumn[
  \maketitle
]

\tableofcontents

%% Section introduction
\section{Introduction}
\subsection{Purpose}
This document has two main purposes:

\begin{itemize}
\item To establish the planning to the whole project (see
  \parref{part:st0-plan}), that means, the common plans to all stages
  of the project (let us call it stage 0 plan).
\item To establish the planning of the current stage 1 (see
  \parref{part:st1-plan}).
\end{itemize}

\subsection{Planning guidance}
This planning has been guided by the next list of documents:

\begin{description}
  \item[\pmbok] This guide is the starting point to planning the
    whole project. It is used here as a biligraphy about project
    management and general guidance about the managerial work to carry
    out.
  \item[IEEE Std 1074-1997] This standard is used as a information
    source to find the activities to carry out in this project.
  \item[IEEE Std 1058-1998] This standard is used as a design model
    to derivate this document.
\end{description}

\subsection{Evolution of the plan}
If there are changes in the project plan, it must to add a history
table at the beginning of this document (just below the
title). Otherwise, this table isn't necessary.

It must to add in this table a row for the first version and for each
update. It must to inform in this table about the new version number
and a short description about the changes made. The format of the
version number must be $x.y$, where $x$ denotes version and $y$
subversion. There are no specifications about when it has to made a
change of version or subversion. This depends on the impact of the
changes made. The top row must be the latest updating, and the bottom
row the initial version, in chronological order.

\part{\favpl}
\label{part:st0-plan}

%% Section overview
\section{Overview}
\subsection{Purpose}
The purpose of this project plan is to establish the planning of the
work to carry out during the whole project. That means, to enumerate
the common activities to perform throughout all stages of the project,
to design a schedule model for any stage (which must be used in each
stage to schedule it fully, adding the stage's specific work), and to
define plans to achieve succesfully the goals described in the \favp.

\subsection{Dependences}
This document constitute a extension of the \favc. The clauses
described here depends thus on the clauses described in the \favc.

%% Section managerial process plans
\section{Managerial process plans}
\subsection{Planning}
\subsubsection{Estimation plan}

\subsection{Executing}
\subsubsection{Work activities}
\subsubsection{Schedule allocation}
\subsubsection{Resource allocation}

\subsection{Control and monitoring}
\subsubsection{Schedule control plan}
\subsubsection{Reporting plan}
\subsubsection{Verification and validation plan}

\subsection{Risk management plan}

\subsection{Closeout plan}

% Section technical process plans
\section{Technical process plans}
\subsection{Process model}
\subsection{Methods, tools and technics}
\subsubsection{Programming style}
\subsection{Infraestructure plan}
\subsection{Deployment plan}

\part{\favpl stage 1}
\label{part:st1-plan}

%% Section introduction
\section{Introduction}

\subsection{Purpose}
The purpose of this document is plan the first stage of the
\favp.

\subsection{Dependences}
This document constitute a extension of the \favpl. The clauses
described here depends thus on the clauses described in the \favpl
(see \parref{part:st0-plan}).

\subsection{Deliverables}
This stage 1 has the next deliverables:

\begin{description}
\item[\fav v0.1] To build the \fav version 0.1, that includes to design a
  \fav programming language which manages algorithm working with
  numeric data types, its operations, functions and recursion and
  control flow structures (sequence, iteration and selection
  structures); and to design a default animation for each of this
  language features.
\item[Documentation] To record this stage and to publish on the
  Internet this information, that includes the registered time for
  each realized activity, the features of the achieved version of the
  \fav and its design.
\end{description}

\section{Managerial process plans}
\subsection{Executing}
\subsubsection{Work activities}
\subsubsection{Schedule allocation}
\subsubsection{Resource allocation}

\section{Technical process plans}
\subsection{Process model}
\subsection{Methods, tools and technics}

\end{document}
