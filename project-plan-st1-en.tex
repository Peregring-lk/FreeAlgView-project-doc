\documentclass[twocolumn, 9pt]{extarticle}

\usepackage[english]{./fav-project-headers/fav-headers}

\title{\fav: \favpl stage 1}
\author{Aarón Bueno Villares \textit{$<$abv150ci@gmail.com$>$}}
\datelicense{October 21, 2012}

\setcounter{tocdepth}{2}

\begin{document}

\twocolumn[
  \maketitle
]

\tableofcontents
\listoffigures
\listoftables

%% Section introduction
\section{Introduction}
\subsection{Purpose}
This document has two main purposes:

\begin{itemize}
\item To establish the planning to the whole project (see
  \parref{part:st0-plan}), that means, the work framework for all
  stages of the project (let us call it stage 0 plan).
\item To establish the planning of the current stage 1 (see
  \parref{part:st1-plan}).
\end{itemize}

\subsection{Planning guidance}
This planning has been guided by the following list of documents:

\begin{description}
  \item[\pmbok] This guide is the starting point to planning the
    whole project. It is used here as a bibliography about project
    management and general guidance about the managerial work to carry
    out.
  \item[IEEE Std 1074-1997] This standard is used as a information
    source to find the activities to carry out in this project.
\end{description}

This project and its plans will not strictly adhered to these
documents. These documents will be used only as general guidance and
source of knowledge.

\part{\favpl}
\label{part:st0-plan}

%% Section overview
\section{Overview}
\subsection{Purpose}
The purpose of this project plan is to establish the planning of the
work to carry out during the whole project. That means, to establish
the work to carry out throughout all stages of the project and plans
on how to carry out this work to achieve succesfully the goals
described in the \favp.

\subsection{Dependences}
This document constitute a extension of the \favc. The clauses
described here depends thus on the clauses described in the \favc.

\subsection{Planning method}
\label{ssec:planning-method}
The method used to plan the project is as follows:

\begin{description}
\item[Identifing risks] To identify all situations which could make the
  project fails.
\item[Identifing at-risk scenes] To identify causes producing these risks.
\item[Identifing requirements] To identify project and product
  requirements to make them resistant to this at-risks scenes.
\item[Designing ``aid plans''] To design plans for
  requirements invisible to the product, that means, those which isn't
  built \textit{inside} the product, but only creating a quality of
  it. These type of requirements require an invisible continuous
  effort to carry out throughout the project.
\end{description}

Any plans will be just now specified, while other plans will be
delegated to later stages of the project, when they are more
meaningful. It is necessary thus to stablish first a dependence
diagram between the activities to detect when it is necessary
develop those plans.

\subsection{Rol of the stage plans}
This project plan defines fundamentally the common work to realice
during all stages of the project. These work have in common that they
relate all to work to non-functional purposes, that means, they are
works to ensure \textit{quality}, but not to develop the functional
requirements of the product.

Thus, the rol of the stage plans is define the work to carry out to
achieve the pure functional requirements specified in the \favc, and
the work to accomplish in a stage is the sum of the functional work of
the stage and the quality work specified in this project plan.

Exceptions are the plans delegated to later project stages, that
are works with quality purposes but defined in concrete stage
plans. Also it is possible in concrete stages to make changes or
extensions in previosly defined plans if this is neccesary.

%\subsection{Changes respect to project managemet}

\section{Quality requirements}
\subsection{Risk identification}
The identified risks, and its causes, in order of priority and dependency
each other, are the following:

\begin{labelist}
\item The project fails.
  \begin{labelist}
    \item Defined objetives not accomplish due to lack of time.
    \item Loss of time due to code rewritting by removal of
      libraries.
  \end{labelist}
\item The product fails.
  \begin{labelist}
    \item Nobody knows the product.
    \item The product doesn't attract new users or developers.
      \begin{labelist}
        \item The information about the product isn't well-defined or
          inaccessible.
        \item Newcommers can't get past the initial obstacle of
          unfamiliarity.
        \item The product doesn't work in the most common plataforms.
        \item The product installation isn't easy.
        \item The product use isn't suitable for newcommers.
        \item The product doesn't work properly.
        \item Developers can't colaborate easily.
      \end{labelist}
    \item Users stop using the product.
      \begin{labelist}
        \item The product doesn't progress.
      \end{labelist}
    \item Developers stop colaborating.
      \begin{labelist}
        \item It is difficult increment the product.
        \item The develop is incoherent.
        \item Lack of new needs, ideas or error reports.
        \item There isn't continuous motivation.
      \end{labelist}
  \end{labelist}
\item Actions to avoid project or product failings, fail.
  \begin{labelist}
    \item The proposed actions aren't correct solutions for this
      problems.
  \end{labelist}
\end{labelist}

\subsection{Requirements}
Only a group of risks are considered in this plan. These risks are
rejected because there is no reason to consider them now. Remainders
risks must be controlled in other stages. Some risks are related or
depend each other, and can be avoided with a only requirement or
plan.

The risks rejected here are the followings: \labelid{B.1},
\labelid{B.2.\{1, 2, 4, 7\}}, \labelid{B.3}, \labelid{B.4.\{2, 3,
  4\}}.

\subsubsection{Requirements for the product}
\label{sssec:prodreq}

\begin{labeledpars}{ProdReq}
  \labeledpar{Documentation for developers} The product source code must be
  written in accordance with readability criteria. The product must
  have a development manual and a code documentation page, both
  freely-available on the Internet.
  \labeledpar{Design} The product source code must be designed in
  accordance with extensibility criteria. The product must be
  multiplataform.
  \labeledpar{GUI} To design an interface in accordance with
  usability criteria. The GUI must be designed for
  internationalization.
\end{labeledpars}

\subsubsection{Requirements for the project}
\begin{labeledpars}{ProjReq}

  \labeledpar{Development plan} The project must have a development
  plan in order to estimate and control better the progress and the
  required time of the project.

  \labeledpar{Importation plan} The project must have an importation
  plan in order to ensure the adjustment of libraries to the product
  needs.

  \labeledpar{Investigation plan} The project must have an
  investigations plan in order to ensure the quality of designs and
  decisions made.

  \labeledpar{Monitoring plan} The project must have a monitoring plan
  in order to improve time estimations.

  \labeledpar{Testing plan} The project must have a testing plan
  in order to ensure the quality of design and implementation of the
  product.

  \labeledpar{Integrated project plan} The project must have an integrated
  project plan in order to know better and estimate the total work to
  carry out.
\end{labeledpars}

\subsection{Risks-requirements matrix}
\label{ssec:rr-matrix}

In this place a matrix relating risks and requirements is shown. The
purpose of this traceability matrix is to detect the source risk of
each requirement. Some risks could need more than one requirement to
avoid it. Similarly, some requirements could concern to more than one
risk. Table \numref{tab:rr-matrix} shows this traceability matrix.

\begin{table*}[th!]
  \caption{Risk-requirements traceability matrix}
  \label{tab:rr-matrix}
  \centering
  \begin{tabular}{|c||c|c|c|c|c|c|c|}
    \hline
    & \mlabelid{A.1} & \mlabelid{A.2} & \mlabelid{B.2.3} &
    \mlabelid{B.2.5} & \mlabelid{B.2.6} & \mlabelid{B.4.1} &
    \mlabelid{C.1} \\
    \hline
    \mlabelid{ProdReq1} & & & & & & \checkmark &\\
    \hline
    \mlabelid{ProdReq2} & & & \checkmark & & & \checkmark &\\
    \hline
    \mlabelid{ProdReq3} & & & & \checkmark & & &\\
    \hline
    \mlabelid{ProjReq1} & \checkmark & & & & & &\\
    \hline
    \mlabelid{ProjReq2} & & \checkmark & & & & &\\
    \hline
    \mlabelid{ProjReq3} & & & & & & & \checkmark\\
    \hline
    \mlabelid{ProjReq4} & \checkmark & & & & & &\\
    \hline
    \mlabelid{ProjReq5} & & & & & \checkmark & & \\
    \hline
    \mlabelid{ProjReq6} & \checkmark & & & & & &\\
    \hline
  \end{tabular}
\end{table*}

\section{Aid plans}
\label{sec:aid-plans}

\subsection{Development plan}
\label{ssec:development-plan}

For simplifying the work to accomplish and improving its measurement,
a development plan and a monitoring plan are established. This clause
describes only the development plan.

\subsubsection{Stages definition}
\label{sssec:stages-definition}
The whole work must be divided in stages. Each stage generates a
subset of the \fav software features and a subset of the \fav
community features. The last stage generates a final version with all
of its features and meeting all requirements and constrains defined
in the \favc and in \parref{sssec:prodreq}.

Each stage must identify the features to accomplish, the new risk to
consider, the requirements to meet, and to estimate the neccesary
time to carry out this work. Identification of new risks could imply
an subsequent investigation in order to identify requirements avoiding
these risks.

\subsubsection{Software release numbering system}
\label{sssec:release}
The actual software state at the end of the stage $x$ must be release
with the software version number $0.x$. For example, at the end of the first
stage, the version number is $0.1$, and at the end of the second
stage, $0.2$. However, the last stage produces the \fav version
$1.0$.

\subsubsection{Documentation}
Each stage must be documented in three forms; other plans add new
related clauses
(\parref{sssec:documentation-importations,ssec:investigation-plan,ssec:monitoring-plan,ssec:testing-plan}):

\paragraph{Stage plans}
\label{par:stage-plans}
Each stage must have its own stage plan. In this stage plan the
selected features to accomplish and the requirements to meet must
be enumerated. If new risks are considered, these must be specified in
the project plan and the risks-requirements matrix
(\parref{ssec:rr-matrix}) must be properly extended and
showed. Finally, the stage timeline must be depicted after estimating
the work time.

\paragraph{Blog}
\label{par:blog}
In the project blog, each stage plan must be uploaded and, at the end
of each stage, each release of a new version (\parref{sssec:release})
must be announced, with the description of achievements and video
samples showing the software running.

\paragraph{Software source code repository}
\label{par:software-source-code-repository}
In the software source code repository, a new tag registering the new
version must be inserted. This tag must be well-described with
information about its actual features and limitations. Within each
stage, changes made on the code must be commited together with good
descriptions in order to record them adequatelly. These commits must
be of use to mark easily borders between groups of conceptually
different changes (not commits with changes realized for
very different purposes), so that a short and precise description
summarize clearly the nature of the commit.

\subsection{Importation plan}
\label{ssec:importation-plan}
This importation plan contains the requirements and policies about the
use of imported software, mainly libraries. This plan is related to
the investigation plan (\parref{ssec:investigation-plan}) and the
testing plan (\parref{ssec:testing-plan}).

\subsubsection{Imported software requirements}
\label{sssec:imported-software-requirements}
All imported software must be libre under GPL-compatible
licenses and multiplataform. All imported software must have an active
development and a good documentation. All imported software must be
selected in order to facilitate the development, and they must be thus
well studied before importing it in the \fav development.

\subsubsection{Imported software testing}
\label{sssec:imported-software-testing}
In order to avoid selecting software which can later delay the
development, a investigation activity must be carried out each time
that a new library is needed to solve a new implementation problem,
searching libraries in line with actual needs and meeting the
described requirements (see above
\parref{sssec:imported-software-requirements}). Qt libraries are free
of this testing process, due to the Qt library usage is imposed in the
\favc.

A posterior testing process must achieved including the creation of
``Hello world'' programs\footnote{Short source code files with a minimal
  example of use.} using that library, and next, more extense programs
with examples of use of all features required for this library, in
order to testing the user-friendliness of the library (that includes if a
good documentation is available) and adequacy of the library features
to the required needs.

As described in the investigation plan
(\parref{ssec:investigation-plan}), each investigation must be
documented to record the usage experience with this library, as well
as to inform readers about the library and the pros and cons of its
choice.

\subsubsection{Documentation of importations}
\label{sssec:documentation-importations}
For each tested library a entry in the \fav blog must be added in
order to document the realized work and to record the usage
experience with this library, as well as to inform readers about the
library and the pros and cons of its choice.

\subsection{Investigation plan}
\label{ssec:investigation-plan}
Each time an important new decision must be taken to meet a
requirement (for example, in order to design a GUI with usability
features or to implement an algorithm to solve a difficult concrete
problem), an investigation must be carried out to ensure the quality
of the decisions made. For each investigation a entry in the \fav blog
must be added in order to document the realized work and to inform
readers about adquired knowledge and the decisions made.

Other investigations are described in
\parref{sssec:stages-definition,sssec:imported-software-testing}.

\subsection{Monitoring plan}
\label{ssec:monitoring-plan}
The time taken for each project activity must be measure in order to
have a better control about the project progress and to improve the
estimations in future stages. Each measurement must be registered in
any form. At the end of each stage, a summary about the activities
performed in the stage and the time taken in each activity must be
posted to the \favp blog.

\subsection{Testing plan}
\label{ssec:testing-plan}
Each stage must have its own testing activity, specially at the end of
the stage. These tests must be carry out in software in order to
ensure the quality of the implementation. These tests must be achieved
just for functional and performance testing purposes:
functional requirements, stress, endurance,
robutness, reliability, regression, volume and so on. Other
non-functional tests as extensibility, usability or escalability tests
aren't obligatory to testing, because they are properties hard to test
and the investigation plan (\parref{ssec:investigation-plan}) is the
means to get this properties in the design phase.

In order to perform good tests a previous investigation, and
optionally a formal design, must be done, except trivial tests. The
conclusion of this investigations must be documented in the \fav
blog.

\section{Integrated project activities}
\label{sec:integrated-activities}
In this section will be defined a integrated list of activities to
carry out during the project based on the plans described above
(\parref{sec:aid-plans}) and the resultant project/development life
cycle.

\favp is divided in stages (4 as maximum) as described in the
\favc. Each stage has two phases: a planning and a execution
phase. Each phase has a group of activities to
accomplish. This activities are extrated from the aid plans
(\parref{sec:aid-plans}) and descomposed in tasks. A section is
dedicated for each identified activity as follows. Tasks are works to
realize obligatorily and they can be of use for estimations, but only
activities must be considered to document plannings/chronograms in
stage plans.

The relationships between aid plans and activities is shown in the
activities/aid plans traceability matrix. While aid plans classifies
work in according to its purpose, activities groups work by similarity
(called more properly tasks). Table \numref{tab:aa-matrix} shows this
traceability matrx.

\begin{table*}[th!]
  \caption{Aid plans-activities traceability matrix}
  \label{tab:aa-matrix}
  \centering
  \begin{tabular}{|c||c|c|c|c|c|}
    \hline
    & \scriptsize{Development plan} & \scriptsize{Importation plan} & \scriptsize{Implementation plan} &
    \scriptsize{Monitoring plan} & \scriptsize{Testing plan}\\
    \hline
    \scriptsize{Identify requirements} & \checkmark & & & &\\
    \hline
    \scriptsize{Estimate time} & \checkmark & & & &\\
    \hline
    \scriptsize{Edit stage plan} & \checkmark & & & &\\
    \hline
    \scriptsize{Import software} & & \checkmark & & &\\
    \hline
    \scriptsize{Develop} & \checkmark & & & &\\
    \hline
    \scriptsize{Investigate} & & & \checkmark & &\\
    \hline
    \scriptsize{Test product} & & & & & \checkmark\\
    \hline
    \scriptsize{Monitor time} & & & & \checkmark &\\
    \hline
    \scriptsize{Document} & \checkmark & \checkmark & \checkmark &
    \checkmark & \checkmark\\
    \hline
  \end{tabular}
\end{table*}

\subsection{Planning phase activities}
\label{ssec:planning-phase}
This phase has the following activities:

\begin{itemize}
\item Identify requirements (\parref{sssec:identify-requirements}).
\item Estimate time (\parref{sssec:estimate-time}).
\item Edit stage plan (\parref{sssec:edit-stage-plan}).
\end{itemize}

\subsubsection{Identify requirements}
\label{sssec:identify-requirements}
This activity involves to establish the requirements (concrete work)
to accomplish/meet throughout the stage. The following tasks
are described (described in its natural order):

\begin{description}
  \item[Select new risks] To select new risks (risks not considered
    before) to deal with throughout the stage.
  \item[Investigate solutions] In case of selecting new risks, to
    investigate its causes and its solutions in order to choose good
    solutions to these risks.
  \item[Select requirements] To select which requirements to
    consider/meet during the stage. This include requirements to
    avoid risks (as described in \parref{ssec:planning-method}) and
    requirements to get goals (the outcomes of the project, see
    \fav and \parref{sssec:develop}).
  \item[Design new plans] New risks and its corresponding requirements
    could need to design additional aid plans to accomplish/meet
    this requirements.
\end{description}

\subsubsection{Estimate time}
\label{sssec:estimate-time}
This activity has as outcome the stage chronogram: the set of
activities to carry out throughout the stage and the planned time to
accomplish these activities. This estimation must be perfomed based on
previous time measurements. No estimation methodologies or processes
are imposed. Descomposition of work in activities/tasks (the purpose
of this section) could facilitate estimations.

\subsubsection{Edit stage plan}
\label{sssec:edit-stage-plan}
This activity groups in one document all information about the work to
carry out throghout stage, which includes risks, requirements, aid
plans and estimations.

\subsection{Execution phase activities}
\label{ssec:execution-phase}
This phase has the following activities:

\begin{itemize}
  \item Import software (\parref{sssec:import-software}).
  \item Develop (\parref{sssec:develop}).
  \item Investigate (\parref{sssec:investigate}).
  \item Test product (\parref{sssec:test-product}).
  \item Monitor time (\parref{sssec:monitor-time}).
  \item Document (\parref{sssec:document}).
\end{itemize}

\subsubsection{Import software}
\label{sssec:import-software}
This activity performs the work described in the importation plan
(\parref{ssec:importation-plan}), except to document test
investigations, moved to the document activity
(\parref{sssec:document}). The following tasks are described:

\begin{description}
  \item[Investigate libraries] To investigate the available libraries
    with the demanded requirements for importations
    (\parref{sssec:imported-software-requirements}), and to choose the
    most suitable/s importations.
  \item[Test libraries] Execution of test as described in
    the second paragraph of \parref{sssec:imported-software-testing}.
\end{description}

\subsubsection{Develop}
\label{sssec:develop}
This activity is the stage nucleus. This consists of pure development
to attain the \fav goals: \fav software and its community (see
\favc). While project requirements are meeting by means of aid
plans, product requirements are meeted most of them with
development (aid plans can meet indirectly some product
requirements). This activity depends on the selected requirements in
the specific stage in order to descompose its tasks. Descomposing
tasks in each stage is a key factor for a good stage definition and
estimation.

\subsubsection{Investigate}
\label{sssec:investigate}
This activity contains any investigation task performed in order to
search development solutions. This activity includes only
investigation for develop purposes. It excludes
investigations for searching requirements, testing the product and its
documentation, considered respectectively in the identify requirements
(\parref{sssec:identify-requirements}), test product
(\parref{sssec:test-product}) and document (\parref{sssec:document})
activities.

\subsubsection{Test product}
\label{sssec:test-product}
This activity performs the quality assurance of the product together
with the investigate activity (\parref{sssec:investigate}). Its work
is described in the testing plan (\parref{ssec:testing-plan}). The
documentation of this tests is a tasks moved to the document activity
(\parref{sssec:document}). The following tasks are described:

\begin{description}
  \item[Investigate tests] To investigate the most suitable methods to
    effectively test the product.
  \item[Design tests] This task consists of designing formally tests
    as a separate work before its application. Described in
    \parref{ssec:testing-plan} as a optional work.
\end{description}

\subsubsection{Monitor time}
\label{sssec:monitor-time}
This activity performs the quality improvement of estimation
throghout stages and the control of the actual realized work. This
activity comes from the monitoring plan
(\parref{ssec:monitoring-plan}). It concerns only measuring the
needed time in the realization of each activity. The documentation of
this monitoring is moved to the document activity
(\parref{sssec:document}).

\subsubsection{Document}
\label{sssec:document}
This activity joins different documentation tasks scattered over
different aid plans. The following tasks are described:

\begin{description}
  \item[Document stage beginnings] To add entries in the \favp blog
    announcing each new stage and upload in it new stage plans, as
    described in the documentation plan (\parref{par:blog}).
  \item[Document importations] To add entries in the \favp blog
    describing each new importation activity performed
    (\parref{sssec:import-software}).
  \item[Document investigations] To add entries in the \favp blog describing
    each new investigation activity performed (\parref{sssec:investigate}).
  \item[Document tests] To add entries in the \favp blog describing
    each new testing activity performed
    (\parref{sssec:test-product}).
  \item[Document stage ends] To add entries in the \favp blog
    announcing the end of each stage, informing about the achieved
    features in the stage and its monitoring summaries, and to update
    source code repository with suitable tags, as described in the
    documentation plan
    (\parref{par:blog,par:software-source-code-repository}).
\end{description}

\subsection{Project development life cycle}
\label{ssec:project-development-life-cycle}
In summary, a Gantt diagram depicting the nature of the project
integrating activities and its dependences is shown. The
representation of execution group of activities is simplified in
order to facilitate understanding and a better illustration of
the relations between activities. This execution actitivies are not
carried out lineally (as shown in the Gantt diagram), but iteratively,
distributing the whole work in smaller packages achieved
progressively, each package containing its own monitoring,
documentation, investigation, importation, development and testing
activities. The project development life cycle of the \favp, joining
all this project features (stage division with its planning process
and its iterative execution process) follows the iterative and
incremental model of development. Figure \numref{fig:life-cycle}
depicts the project development life cycle.

\begin{figure*}[th!]

  \caption{Project development life cycle}
  \label{fig:life-cycle}

  \begin{center}
    \begin{ganttchart}[vgrid, hgrid,
        y unit title=0.7cm,
        y unit chart=0.5cm,
        canvas/.style={draw=none},
        title/.style={fill=blue!20, draw=none, rounded corners=2mm},
        title label font=\scriptsize,
        milestone label font=\scriptsize,
        group label
        font={\scriptsize\bfseries\fcolorbox{white}{white}},
        group label inline anchor/.style={yshift=0pt},
        bar label font=\scriptsize,
        bar/.style={rounded corners=2pt, fill=green!20!white},
        supportbar/.style={rounded corners=2pt, fill=red!20!white},
        group right shift=0,
        group top shift=.6,
        group height=.1,
        group peaks={}{}{.2}
        stagex1=3,
        stagex1 rule/.style={draw=black!64, dash pattern=on 3.5pt off
          4.5pt, line width=1.5pt}]{26}
      \gantttitle{Stage $x-1$}{3}
      \gantttitle{Stage $x$}{20}
      \gantttitle{Stage $x+1$}{3}\\

      \ganttmilestone{\fav $0.x-1$}{2}\\

      \ganttlinkedgroup[name=plan, inline]{Planning}{4}{8}\\

      \ganttbar{Identify requirements}{4}{5}\\
      \ganttlinkedbar{Estimate time}{6}{6}\\
      \ganttlinkedbar{Edit stage plan}{7}{8}\\

      \ganttlinkedgroup[name=exe, inline]{Execution}{9}{22}\\
      \ganttbar{Monitor time}{9}{22}\\
      \ganttbar{Document}{9}{22}\\
      \ganttbar{Investigate}{9}{22}\\
      \ganttbar{Import software}{9}{10}\\
      \ganttlinkedbar{Develop}{11}{21}\\
      \ganttlinkedbar{Test product}{22}{23}\\

      \ganttlinkedmilestone{\fav $0.x$}{23}\\
    \end{ganttchart}

  \end{center}

\end{figure*}

\part[\favpl st. 1]{\favpl \\ stage 1}
\label{part:st1-plan}

\section{Overview}

\subsection{Purpose}
The purpose of this document is plan the first stage of the
\favp.

\subsection{Dependences}
This document constitute a extension of the \favpl. The clauses
described here depends thus on the clauses described in the \favpl
(see \parref{part:st0-plan}).

\subsection{Deliverables}
This stage 1 has the following deliverables:

\begin{description}
\item[\fav v0.1] To build the \fav version 0.1, that includes to design a
  \fav programming language which manages algorithm working with
  numeric data types, its operations, functions, recursion and
  control flow structures (sequence, iteration and selection
  structures); and to design a default animation for each of this
  language features.
\end{description}

\section{Work detach}
\subsection{Risks, requirements and aid plans}
\label{ssec:r-r-aid-plans}
The risks selected for this stage are the same selected in the \favpl
(\parref{part:st0-plan}), except the risk \labelid{B.2.5} due to the
product requirement \labelid{ProdReq3} is rejected in this stage (see
table \numref{tab:rr-matrix}). All aid plans defined in the \favpl
(\parref{sec:aid-plans}) are inherited here without changes.

In respect of the goals described in the \favc, this stage considers
only the goal \textbf{G0} (extracted from the \favc):

\begin{quote}
  \textit{To build an application improving algorithm learning by means of
  graphical animations of the process execution of any algorithm.}
\end{quote}

From this goal, this stage defines a subset of this features by means
of the following additional requirements, giving rise the \fav version
0.1:

\def\pronounciationfootnote{%
    For convenience, the vowel \textit{u} replaces
    \textit{v} to pronounce \favpp. The valid \favpp pronunciation in
    English is thus \textit{fau plus plus}
    (\textipa{[""fa\textsubarch{U}.""pl2s."pl2s]}), while in Spanish they
    are \textit{fau pe pe} (\textipa{[""f\=*a\textsubarch{u}.""p\|`e."p\|`e]}), \textit{fau plus
      plus} (\textipa{[""f\=*a\textsubarch{u}.""plus."plus]}) and
    \textit{fau más más}
    (\textipa{[""f\=*a\textsubarch{u}.""m\=*as."m\=*as]}).
}

\begin{labeledpars}[4]{ProdReq}
  \labeledpar{\fav language v0.1} \fav v0.1 must have defined a \fav
  programming language, called \favpp\twofootnotes{\fav \textit{plus plus} in
    honour to \textit{C++}, the development language of
    \fav.}{\pronounciationfootnote} v0.1, which
  can manage operations with any primitive type (classical types as
  integers, floats, characters, strings, and so on), functions with
  any (finite) arity, recursion, and control flow statements (sequence,
  iteration and selection structures).
  \labeledpar{GUI v0.1} \fav v0.1 must have a GUI by which the user can
  introduce an algorithm written in \favpp (v0.1) and the application
  visualices the execution of this algorithms by means of a default
  (not customizable) animation.
\end{labeledpars}

\subsection{Develop tasks}
\label{ssec:develop-tasks}
Two subactivities are identified:

\begin{itemize}
\item Design (\parref{sssec:design}).
\item Implement (\parref{sssec:implement}).
\end{itemize}

\subsubsection{Design}
\label{sssec:design}
The main objetive of the design subactivity is to design the
default animation with the requirements described in the risks,
requirements and aid plans section (\parref{ssec:r-r-aid-plans}). With
this designed default animation as a starting point, language
features, GUI, core, etcetera, must be designed. Thus, some tasks are
identified for this first stage:

\begin{itemize}
\item Design \fav default animation v0.1.
\item Design \favpp v0.1.
\item Design \fav system architecture v0.1.
\item Design \fav GUI v0.1.
\item Design \fav system core v0.1.
\end{itemize}

\subsubsection{Implement}
\label{sssec:implement}
This tasks is responsible for implement the designs made in the design
subactivity. Respectivelly, the next tasks are identified:

\begin{itemize}
\item Implement \fav system architecture v0.1.
\item Implement \favpp parser v0.1.
\item Implement \fav system core v0.1.
\item Implement \fav default animation v0.1.
\item Implement \fav GUI v0.1.
\end{itemize}

\subsection{Investigation tasks}
\label{ssec:investigation-tasks}
From the tasks described in the develop tasks section
(\parref{ssec:develop-tasks}), the investigation activity is divided
in two analogous tasks:

\begin{description}
\item[Investigate design issues] Investigations related to get good
  design decisions.
\item[Investigate technical issues] Investigations related to solve
  implementation problems or other technical problems.
\end{description}

\section{Stage development model}
\label{sec:stage-development-life-cycle}
In this section, the \favp stage 1 model of development and its
temporal planification will be established. The planning phase is
excluded of this development model due to this document is itself the
end of the planning phase.

\subsection{Stage model}
\label{ssec:stage-model}
The developt activity of this substage is divided in
four subphases, each one of them aim to meet features in increasing
complexity order. Subphases and its purposes are the followings:

\begin{description}
  \item[Subphase 1.1] This subphase aims to implement an default
    animation of values of algorithms working only with numbers. It
    exludes animation of operations, recursion, control flow
    statements or other non-numeric types. In this subphase is also
    implemented the GUI and the parser for a preliminary version of
    \favpp. The outcome of this subphase is \fav v0.0.6.
  \item[Subphase 1.2] This subphase adds to \fav v0.0.6 default
    animation of operations. Its outcome is \fav v0.0.7.
  \item[Subphase 1.3] This subphase adds to \fav v0.0.7 values and
    operations for any primitive type, that includes to increment
    \favpp to support other primitive types. Its outcome is \fav
    v0.0.8.
  \item[Subphase 1.4] This subphase adds to \fav v0.0.8 animation of
    functions, recursion and control flow statements, that includes to
    increment \favpp to support functions and recursion. This subphase
    achieve the requirements defined for this stage
    (\parref{ssec:r-r-aid-plans}).
\end{description}

Before subphase 1.1 is the import software activity, in which all
required libraries for the whole phase must be identified,
investigated and tested. After subphase 1.3 is the test product
activity, in which the product must successfully pass all quality
tests. After completion of this activity, the \favp stage 1 concludes
and \fav v0.1 is launched.

\subsection{Stage estimation}
\label{ssec:estimation}
A 6-weeks term for the stage 1 is proposed. Let us assume \textit{an average}
of 5-hours per working day, seven days at week. The estimated time taken
for each activity are thus the followings:

\begin{description}
\item[Monitor time] This activity take an insignificant time to carry
  out it. Its estimation time is 0 hours.
\item[Document] 10 hours.
\item[Investigate] 60 hours.
\item[Import software] 10 hours.
\item[Develop] 120 hours.
  \begin{description}
  \item[Design] 20 hours.
  \item[Implement] 100 hours.
  \end{description}
\item[Test product] 10 hours.
\end{description}

Due to the three first activities are distributed along all the stage,
to represent milestones it is neccesary to distribute its time along
the remaining time. Let us assume the document and investigate
activities begin after the import software activity. Thus, it is
neccesary to distribute 70 hours (document plus investigate time)
throughout the remaining 130 hours (develop plus test product
time): \[ 130 * p = 130 + 70 = 200 \Rightarrow p = \frac{200}{130} =
1.5385 \] Aplying this expansion coefficient $p$ to develop and test
product activities, and to round them (as multiples of 5 hours), the
next measurements are getting:

\begin{description}
  \item[Develop] $120 * p = 184.65 \Rightarrow 185$ hours.
    \begin{description}
      \item[Design] $20 * p = 30.769 \Rightarrow 30$ hours.
      \item[Implement] $100 * p = 153.85 \Rightarrow 155$ hours.
    \end{description}
  \item[Test product] $10 * p = 15.385 \Rightarrow 15$ hours.
\end{description}

In turn, the develop activity must be distribute in the four subphases
of this stage (\parref{ssec:stage-model}). Depending of the estimate
difficulty for each subphase, the following percentages of time for
each subphase and its repectives times are the followings:

\begin{description}
\item[Subphase 1.1] 35\% of the develop time: $185 * 0.35 = 64.75
  \Rightarrow 65$ hours.
  \begin{description}
    \item[Design] $30 * 0.35 = 10.5 \Rightarrow 10$ hours.
    \item[Implement] $155 * 0.35 = 54.25 \Rightarrow 55$ hours.
  \end{description}
\item[Subphase 1.2] 25\% of the develop time: $185 * 0.25 = 46.25
  \Rightarrow 45$ hours.
  \begin{description}
    \item[Design] $30 * 0.25 = 7.25 \Rightarrow 5$ hours.
    \item[Implement] $155 * 0.25 = 38.75 \Rightarrow 40$ hours.
  \end{description}
\item[Subphase 1.3] 15\% of the develop time: $185 * 0.15 = 27.75
  \Rightarrow 25$ hours.
  \begin{description}
  \item[Design] $30 * 0.15 = 4.5 \Rightarrow 5$ hours.
  \item[Implement] $155 * 0.15 = 23.25 \Rightarrow 20$ hours.
  \end{description}
\item[Subphase 1.4] 25\% of the develop time, the same percentage as
  subphase 1.2: 45 hours.
  \begin{description}
    \item[Design] 5 hours.
    \item[Implement] 40 hours.
  \end{description}
\end{description}

With this estimations and adjustements, the Gantt diagram with the
complete timeline and milestones is depicted in the table
\numref{fig:stage1-life-cycle}.

\begin{figure*}[th!]

  \caption{Stage 1 development life cycle (execution phase)}
  \label{fig:stage1-life-cycle}

  \begin{leftbar}
    \begin{center}
      \begin{ganttchart}[vgrid, hgrid,
          x unit = 0.34cm,
          y unit title=0.7cm,
          y unit chart=0.5cm,
          link bulge=0.6,
          canvas/.style={draw=none},
          title/.style={fill=blue!20, draw=none, rounded corners=2mm},
          title label font=\scriptsize,
          milestone label font=\scriptsize,
          group label
          font={\scriptsize\bfseries\fcolorbox{white}{white}},
          group label inline anchor/.style={yshift=0pt},
          bar label font=\scriptsize,
          bar/.style={rounded corners=2pt, fill=green!20!white},
          supportbar/.style={rounded corners=2pt, fill=red!20!white},
          group right shift=0,
          group top shift=.6,
          group height=.1,
          group peaks={}{}{.2}]{42}
        \gantttitle{Nov.2012}{19}
        \gantttitle{Doc.2012}{23}\\
        \gantttitle{Week 1}{7}
        \gantttitle{Week 2}{7}
        \gantttitle{Week 3}{7}
        \gantttitle{Week 4}{7}
        \gantttitle{Week 5}{7}
        \gantttitle{Week 6}{7}\\
        \gantttitlelist{12,...,30}{1}
        \gantttitlelist{1,...,23}{1}\\

        \ganttgroup[inline]{Execution}{1}{42}\\

        \ganttbar{Monitor time}{1}{42}\\
        \ganttbar{Document}{1}{42}\\
        \ganttbar{Investigate}{1}{42}\\
        \ganttbar{Import software}{1}{2}\\

        \ganttlinkedgroup[inline]{Develop}{3}{39}\\

        \ganttgroup[inline]{\tiny{Subph. 1.1}}{3}{15}\\
        \ganttbar{Design}{3}{4}\\
        \ganttlinkedbar{Implement}{5}{15}\\
        \ganttlinkedmilestone{\fav 0.0.6}{15}\\

        \ganttlinkedgroup[inline]{\tiny{Subph. 1.2}}{16}{24}\\
        \ganttbar{Design}{16}{16}\\
        \ganttlinkedbar{Implement}{17}{24}\\
        \ganttlinkedmilestone{\fav 0.0.7}{24}\\

        \ganttlinkedgroup[inline]{\tiny{Subph. 1.3}}{25}{30}\\
        \ganttbar{Design}{25}{25}\\
        \ganttlinkedbar{Implement}{26}{30}\\
        \ganttlinkedmilestone{\fav 0.0.8}{30}\\

        \ganttlinkedgroup[inline]{\tiny{Subph. 1.4}}{31}{39}\\
        \ganttbar{Design}{31}{31}\\
        \ganttlinkedbar{Implement}{32}{39}\\
        \ganttlinkedmilestone{\fav 0.0.9}{39}\\

        \ganttlinkedbar{Test product}{40}{42}\\
        \ganttlinkedmilestone{\fav 0.1}{42}
      \end{ganttchart}

    \end{center}
  \end{leftbar}
\end{figure*}

\licensesection{Aarón Bueno Villares}{2012}

\end{document}
